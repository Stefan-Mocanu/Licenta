\cleardoublepage
% \phantomsection
% \addcontentsline{toc}{section}{Rezumat}
{\centering \section*{Rezumat} \par}

Acest proiect propune un instrument de tip linie de comandă pentru parsarea și simularea Petri Neturilor, cu suport pentru extensii precum tokenuri colorate, arce inhibitoare, constrângeri temporale și construcții high-level cu placeuri parametrizate. Creat în principal cu scop educațional pentru aprofundarea conceptelor de design al compilatoarelor și modelare formală, instrumentul transformă un limbaj specific într-o reprezentare internă, permițând simularea, generarea de cod și vizualizarea neturilor prin Graphviz. Deși nu include o interfață grafică și nu respectă complet standarde existente precum PNML, proiectul oferă o abordare flexibilă, bazată pe text, potrivită pentru experimente cu modele mari sau complexe. Posibile dezvoltări viitoare includ adăugarea unei interfețe grafice, îmbunătățirea semanticii rețelelor high-level și integrarea cu formate standard.

\bigskip

\bigskip

% \addcontentsline{toc}{section}{Abstract}
{\centering \section*{Abstract} \par}
This project presents a command-line tool for parsing and simulating Petri Nets, with support for various extensions such as colored tokens, inhibitor arcs, time constraints, and high-level constructs with parameterized places. Designed primarily for learning about compiler design and formal models, the tool translates a domain-specific language into an internal representation, allowing simulation, code generation, and visualization through Graphviz. While it lacks a graphical interface and full support for existing standards like PNML, the project offers a flexible, text-based approach suitable for experimenting with large or complex models. Potential future work includes adding graphical support, improving high-level net semantics, and integrating standard formats.