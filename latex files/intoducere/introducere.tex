\documentclass[12pt]{article}
\usepackage{times}\usepackage{setspace}\doublespacing\usepackage[margin=2.5cm]{geometry}
\begin{document}
\section{Introduction}
    \subsection{The type of the thesis and the specific subfield}
    This thesis is a research and software development project. 
    It falls within the field of Petri net theory and compilers, 
    with applications in modeling and simulation of concurrent systems.

    \subsection{General presentation of the topic}

    Petri nets are a mathematical modeling tool used to describe and analyze concurrent, distributed, and parallel systems. 
    They provide a formal way to represent states and transitions, 
    making them useful in verifying properties such as 
    deadlocks, reachability, and synchronization in complex systems.

    Compilers, on the other hand, are essential in transforming high-level representations into executable models, 
    automating the process of simulation and analysis. 
    By combining Petri net theory with compiler techniques, 
    we can create a language that enables users to define Petri nets in a structured manner 
    and automatically generate executable code for simulation. 
    This approach facilitates the modeling, testing and visualization of 
    concurrent systems, making it easier to analyze their behavior.

    \subsection{Purpose and motivation for choosing the topic}

    Modeling concurrent and parallel systems is essential for analyzing distributed processes, 
    formal verification, and their optimization. 
    Petri nets are a powerful tool in this field, and a dedicated language can simplify 
    the definition and testing of models. 
    The motivation for choosing this topic is both the theoretical interest in formal systems 
    and the practical applicability in verifying concurrent software.

    \subsection{Own contribution to the thesis}

    I have developed a description language for Petri nets and a compiler written in Haskell 
    that transforms this language into a Golang executable representation and LaTeX code for visualization. 
    The project combines concepts from formal language theory, compilers, 
    and concurrent system modeling.

    \subsection{Structure of the thesis}
    \begin{enumerate}
        \item Introduction: Presents the context of the thesis, motivation, and personal contribution.
        \item TBA
    \end{enumerate}
\section{Preliminaries}
    \subsection{Scientific and Technological Concepts Underlying the Topic}
    \begin{itemize}
        \item Petri net
        \item Place
        \item Transition
        \item Marking
        \item Firing a transition
        \item Cicle detection
        \item Timed petri nets
        \item Coloured petri nets
        \item Inhibitor Arcs
        \item Highlevel nets
    \end{itemize}
    \subsection{Current State of the Specific Subfield}
    ********TBA
    \subsection{Objectives of the Thesis in Context}
    The main goal of this thesis is to develop a domain-specific language for defining Petri nets, along with a compiler that translates these definitions into executable code for simulation and LaTeX code for visualization. The specific objectives are:
    \begin{itemize}
        \item To design a simple and expressive language for defining Petri nets.
        \item To implement a compiler in Haskell that generates both simulation-ready code and graphical representations.
        \item To compare this approach with existing tools and evaluate its advantages in terms of usability, automation, and flexibility.
    \end{itemize}
    This project aims to bridge the gap between formal modeling techniques and practical implementation, making Petri net analysis more accessible and automated.
\end{document}
