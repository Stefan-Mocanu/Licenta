% !TeX TXS-program:compile = txs:///pdflatex/[--shell-escape]
\documentclass{article}

\usepackage[pdf]{graphviz}
\usepackage{graphicx}

\begin{document}

My graph:

\resizebox{\textwidth}{!}{
\digraph{abc} digraph PetriNet {
  rankdir=LR;
  node [shape=circle, width=0.5, style=filled, fillcolor=lightgray];
  ClientIdle;
  RequestSent;
  Processing;
  ResponseReady;
  ClientReceived;
  ServerIdle;
  node [shape=rectangle, width=0.1, height=0.6, style=filled, fillcolor=black];
  SendRequest [xlabel="SendRequest", label=""];
  ProcessRequest [xlabel="ProcessRequest", label=""];
  SendResponse [xlabel="SendResponse", label=""];
  ReceiveResponse [xlabel="ReceiveResponse", label=""];
  subgraph cluster_N {
    style=filled;
    color=lightgrey;
    node [style=filled,color=white];
    ClientIdle; RequestSent; Processing; ResponseReady; ClientReceived; SendRequest; ProcessRequest; SendResponse; ReceiveResponse;
    label = "N";
  }
  ClientIdle -> SendRequest [label=<<TABLE BORDER="0" CELLBORDER="0" CELLSPACING="0"><TR><TD>{blue:1}</TD></TR><TR><TD><FONT POINT-SIZE="10">inh={blue:1}</FONT></TD></TR></TABLE>>, headport=w];
  SendRequest -> RequestSent [label=<{blue:1}>, tailport=e];
  RequestSent -> ProcessRequest [label=<{blue:1}>, headport=w];
  ServerIdle -> ProcessRequest [label=<{green:1}>, headport=w];
  ProcessRequest -> Processing [label=<{blue:1}>, tailport=e];
  ProcessRequest -> ServerIdle [label=<{green:1}>, tailport=e];
  Processing -> SendResponse [label=<{blue:1}>, headport=w];
  SendResponse -> ResponseReady [label=<{blue:1}>, tailport=e];
  ResponseReady -> ReceiveResponse [label=<{blue:1}>, headport=w];
  ReceiveResponse -> ClientReceived [label=<{blue:1}>, tailport=e];
  ReceiveResponse -> ClientIdle [label=<{blue:1}>, tailport=e];
}
}
\end{document}
