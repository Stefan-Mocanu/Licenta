% !TeX TXS-program:compile = txs:///pdflatex/[--shell-escape]
\documentclass{article}

\usepackage[pdf]{graphviz}
\usepackage{graphicx}

\begin{document}

My graph:

\resizebox{\textwidth}{!}{
\digraph{abc} {
    rankdir=LR;
    node [shape=circle, width=0.5, style=filled, fillcolor=lightgray];
    PF;
    P1;
    P2;
    P3;
    node [shape=rectangle, width=0.1, height=0.6, style=filled, fillcolor=black];
    T2 [xlabel="T2", label=""];
    T1 [xlabel="T1", label=""];
    subgraph cluster_N {
      style=filled;
      color=lightgrey;
      node [style=filled,color=white];
      PF; T2;
      label = "N";
    }
    P1 -> T1 [label=<{blue:4,red:7}>, headport=w];
    P2 -> T1 [label=<{blue:66,red:5,yellow:8}>, headport=w];
    T1 -> P3 [label=<<TABLE BORDER="0" CELLBORDER="0" CELLSPACING="0"><TR><TD>{blue:66,red:5,yellow:8}</TD></TR><TR><TD><FONT POINT-SIZE="10">inh={blue:66,red:5,yellow:8}</FONT></TD></TR></TABLE>>, tailport=e];
    PF -> T2 [label=<<TABLE BORDER="0" CELLBORDER="0" CELLSPACING="0"><TR><TD>{blue:66,red:5,yellow:8}</TD></TR><TR><TD><FONT POINT-SIZE="10">inh={blue:66,red:5,yellow:8}</FONT></TD></TR></TABLE>>, headport=w];
    T2 -> P1 [label=<{blue:66,red:5,yellow:8}>, tailport=e];
  }
}
\end{document}
